%==============================================================================
% Sjabloon onderzoeksvoorstel bachproef
%==============================================================================
% Gebaseerd op document class `hogent-article'
% zie <https://github.com/HoGentTIN/latex-hogent-article>

% Voor een voorstel in het Engels: voeg de documentclass-optie [english] toe.
% Let op: kan enkel na toestemming van de bachelorproefcoördinator!
\documentclass{hogent-article}

% Invoegen bibliografiebestand
\addbibresource{voorstel.bib}

% Informatie over de opleiding, het vak en soort opdracht
\studyprogramme{Professionele bachelor toegepaste informatica}
\course{Bachelorproef}
\assignmenttype{Onderzoeksvoorstel}
% Voor een voorstel in het Engels, haal de volgende 3 regels uit commentaar
% \studyprogramme{Bachelor of applied information technology}
% \course{Bachelor thesis}
% \assignmenttype{Research proposal}

\academicyear{2025-2026} % TODO: pas het academiejaar aan

% TODO: Werktitel
\title{Competitie Framework voor System Administrators Level 1 \& 2 voor z/OS omgevingen}

% TODO: Studentnaam en emailadres invullen
\author{Murat Bugdayci}
\email{murat.bugdayci@student.hogent.be}

% TODO: Medestudent
% Gaat het om een bachelorproef in samenwerking met een student in een andere
% opleiding? Geef dan de naam en emailadres hier
% \author{Yasmine Alaoui (naam opleiding)}
% \email{yasmine.alaoui@student.hogent.be}

% TODO: Geef de co-promotor op
\supervisor[Co-promotor]{A. Kornelis (Bixoft NL,\href{mailto:abe@bixoft.nl}{mailto:abe@bixoft.nl})}
\projectrepo{https://github.com/muratbugdayc/2026bachelorproefMuratBugdayci}
% Binnen welke specialisatierichting uit 3TI situeert dit onderzoek zich?
% Kies uit deze lijst:
%
% - Mobile \& Enterprise development
% - AI \& Data Engineering
% - Functional \& Business Analysis
% - System \& Network Administrator
% - Mainframe Expert
% - Als het onderzoek niet past binnen een van deze domeinen specifieer je deze
%   zelf
%
\specialisation{Mainframe Expert}
\keywords{z/OS, Mainframe, System Administrators}

\begin{document}

\begin{abstract}
	Veel grote bedrijven, zoals banken, overheidsinstellingen en verzekeringsmaatschappijen, blijven in sterke mate afhankelijk van bedrijfskritische IT systemen die draaien op mainframe omgevingen die gebruik maken van z/OS besturingssystemen. Het beheer van deze omgevingen vereist gespecialiseerde kennis en vaardigheden, terwijl de instroom van nieuwe, ervaren system administrators voor z/OS systemen onder druk staat. Hierdoor ontstaat binnen organisaties een groeiende nood aan duidelijke rolafbakening, kennisborging en gestructureerde opleidings en doorgroeipaden.
	Deze bachelorproef richt zich op het ontwikkelen van een competentieframework als proof of concept voor System Administrators Level 1 en Level 2 binnen z/OS omgevingen. Het doel van dit framework is het identificeren en structureren van relevante competenties op het vlak van technische kennis, operationele vaardigheden en professionele verantwoordelijkheden.
	De centrale onderzoeksvraag focust op welke kerncompetenties noodzakelijk zijn om de rol van System Administrator op verschillende ervaringsniveaus effectief te kunnen invullen binnen een z/OS context. De methodologie bestaat uit een literatuurstudie aangevuld met analyse van bestaande rolbeschrijvingen, best practices en industriestandaarden. Op basis hiervan wordt een conceptueel framework opgesteld dat kan dienen als fundament voor verdere verfijning, opleiding en evaluatie.
	Het verwachte resultaat is een gestructureerd en toepasbaar competentieframework namelijk een website dat organisaties of scholen ondersteunt bij kennisoverdracht, talentontwikkeling en continuïteit van mainframebeheer, en dat tegelijkertijd richting geeft aan de ontwikkeling van system administrators.
\end{abstract}


\tableofcontents

% De hoofdtekst van het voorstel zit in een apart bestand, zodat het makkelijk
% kan opgenomen worden in de bijlagen van de bachelorproef zelf.
%---------- Inleiding ---------------------------------------------------------

% TODO: Is dit voorstel gebaseerd op een paper van Research Methods die je
% vorig jaar hebt ingediend? Heb je daarbij eventueel samengewerkt met een
% andere student?
% Zo ja, haal dan de tekst hieronder uit commentaar en pas aan.

%\paragraph{Opmerking}

% Dit voorstel is gebaseerd op het onderzoeksvoorstel dat werd geschreven in het
% kader van het vak Research Methods dat ik (vorig/dit) academiejaar heb
% uitgewerkt (met medesturent VOORNAAM NAAM als mede-auteur).
% 

\section{Inleiding}%
\label{sec:inleiding}

Grote organisaties zoals banken, verzekeringsmaatschappijen en overheidsinstellingen blijven intensief gebruikmaken van mainframe omgevingen, omdat deze systemen uitzonderlijke betrouwbaarheid en continuïteit garanderen voor bedrijfskritische processen. Mainframes kenmerken zich door zeer hoge uptime percentages, tot 99,999~\%, en robuuste beveiligingsmechanismen die essentieel zijn voor financiële en institutionele workloads \autocite{MainframeReliabilityEnterprise}.

Het meest gebruikte besturingssysteem op deze platforms, z/OS, is specifiek ontworpen om een stabiele, veilige en continu beschikbare omgeving te bieden voor kritische toepassingen \autocite{IBMzOSOverview}.

\textcolor{red}{De concrete probleemstelling volgt als:}
\textcolor{red}{Het beheer van z/OS-omgevingen vereist echter sterk gespecialiseerde kennis. System Administrators moeten niet alleen beschikken over diepgaande technische expertise, maar ook over inzicht in operationele processen, beveiliging, automatisering en scripting. Tegelijkertijd staat de instroom van nieuwe, ervaren mainframe profielen onder druk door vergrijzing en een beperkte instroom vanuit het onderwijs. Hierdoor ontstaat binnen organisaties een toenemende nood aan gestructureerde kennisoverdracht en een duidelijke afbakening van rollen, verantwoordelijkheden en technische kennis van z/OS.}

Binnen deze context ontbreekt het in veel organisaties of onderwijsinstellingen aan een eenduidig en gestructureerd competentiekader voor System Administrators op verschillende ervaringsniveaus. Dit bemoeilijkt niet alleen de onboarding en opleiding van nieuwe medewerkers, maar ook de evaluatie, doorgroei en kennisborging binnen teams.

\textcolor{blue}{De centrale onderzoeksvraag van deze bachelorproef luidt als volgt:}

\textcolor{blue}{\textit{“Welke kerncompetenties zijn noodzakelijk om de rol van System Administrator Level~1 en Level~2 effectief te kunnen invullen binnen een z/OS omgeving?”}}

Om deze onderzoeksvraag te beantwoorden, worden de volgende deelvragen onderzocht.

\textcolor{purple}{\textbf{Deelvragen met betrekking tot het probleemdomein:}}
\textcolor{purple}{
	\begin{itemize}
	\item Welke taken en verantwoordelijkheden worden in de praktijk verwacht van een System Administrator binnen een z/OS omgeving?
	\item Welke verschillen bestaan er tussen het profiel van een System Administrator Level~1 en Level~2 in mainframe omgevingen?
	\item Welke knelpunten ervaren organisaties bij het ontbreken van een gestructureerde rol- en competentieafbakening?
\end{itemize}
}

\textcolor{orange}{\textbf{Deelvragen met betrekking tot het oplossingsdomein:}}
\textcolor{orange}{
		\begin{itemize}
		\item Hoe kunnen competenties voor System Administrators binnen z/OS-omgevingen gestructureerd worden in een overzichtelijk framework?
		\item Op welke manier kan een onderscheid tussen Level~1 en Level~2 worden uitgewerkt binnen dit framework?
		\item Hoe kan dit competentieframework op een toegankelijke manier worden voorgesteld via een proof of concept (PoC) namelijk een website?
	\end{itemize}
}

Het doel van dit onderzoek is het ontwikkelen van een gestructureerd en toepasbaar competentieframework voor System Administrators Level~1 en Level~2 binnen z/OS-omgevingen. Dit framework brengt de vereiste competenties in kaart op het vlak van technische kennis, operationele vaardigheden en professionele verantwoordelijkheden.

\textcolor{green}{
	De doelgroep van deze bachelorproef bestaat uit studenten met een focus op mainframe tech\- nologie aan HOGENT of andere onderwijsinstellingen met mainframe-gerelateerde opleidingsonderdelen, evenals uit junior of startende System Administrators die actief zijn binnen organisaties waar gebruik wordt gemaakt van IBM mainframes en/of z/OS.
}


Het concrete eindresultaat van deze bachelorproef is een proof of concept (PoC) in de vorm van een website waarin het competentieframework op een overzichtelijke manier wordt voorgesteld. Dit framework kan dienen als fundament voor verdere verfijning, opleidingstrajecten en evaluatie binnen organisaties die afhankelijk zijn van mainframe-infrastructuur. Het onderzoek wordt als succesvol beschouwd wanneer het framework voldoende duidelijk, herkenbaar en bruikbaar is voor professionals die actief zijn binnen een z/OS context.


%---------- Stand van zaken ---------------------------------------------------

\section{Literatuurstudie}%
\label{sec:literatuurstudie}

Mainframes blijven een essentieel onderdeel van de IT infrastructuur van grote organisaties, met name in sectoren waar betrouwbaarheid, schaalbaarheid en beveiliging cruciaal zijn. Volgens \textcite{MainframeReliabilityEnterprise} onderscheiden mainframes zich van alternatieve platformen door hun hoge beschikbaarheid en robuuste fouttolerantie, wat hen bijzonder geschikt maakt voor bedrijfskritische toepassingen.

Binnen deze omgevingen speelt het besturingssysteem z/OS een centrale rol. z/OS is specifiek ontworpen om grote hoeveelheden transacties gelijktijdig te verwerken en biedt uitgebreide ondersteuning voor beveiliging, workloadmanagement en systeemautomatisering \autocite{IBMzOSOverview}. Deze eigenschappen maken het beheer van z/OS omgevingen complex en kennisintensief.

\subsection{Mainframe omgevingen en z/OS}

Mainframes vormen al decennialang een stabiele pijler binnen de IT-infrastructuur van grote organisaties, in het bijzonder in sectoren waar betrouwbaarheid, schaalbaarheid en beveiliging van cruciaal belang zijn. In tegenstelling tot gedistribueerde of cloudgebaseerde platformen zijn mainframes specifiek ontworpen om grote volumes aan transacties continu en fouttolerant te verwerken. Daarnaast beschrijft IBM Z architectuurdocumentatie de sterke integratie tussen hardware en software als kenmerkend voor mainframe systemen \autocite{IBMZArchitecture}.


Een kerncomponent binnen moderne mainframe omgevingen is het besturingssysteem z/OS. Dit besturingssysteem is ontwikkeld met het oog op hoge beschikbaarheid, geavanceerd workloadmanagement en strikte beveiligingscontrole. Volgens de officiële documentatie van IBM ondersteunt z/OS het gelijktijdig uitvoeren van meerdere bedrijfskritische toepassingen, terwijl het systeem continu beschikbaar blijft voor eindgebruikers \autocite{IBMzOSOverview}.

Het beheer van z/OS omgevingen vereist een hoge mate van technische expertise en inzicht in de onderliggende systeemarchitectuur. Naast het configureren en onderhouden van het besturingssysteem dienen beheerders rekening te houden met aspecten zoals performance optimalisatie, beveiligingsconfiguratie en foutafhandeling. De IBM Redbooks reeks onderstreept dat z/OS systemen gekenmerkt worden door een sterke samenhang tussen hardware, software en operationele processen, wat het beheer ervan fundamenteel verschilt van andere serverplatformen \autocite{IBMRedbookzOS}.

Deze combinatie van complexiteit, kriticiteit en langdurig gebruik binnen organisaties verklaart waarom mainframe omgevingen een gespecialiseerd kennisdomein vormen. De literatuur toont aan dat een gebrek aan gestructureerde kennisoverdracht en duidelijke rolafbakening binnen dergelijke omgevingen een verhoogd risico kan vormen voor de continuïteit van IT diensten, wat het belang van verdere studie naar competenties binnen dit domein onderstreept \autocite{MainframeReliabilityEnterprise}.


\subsection{De rol van de System Administrator}

De rol van de System Administrator omvat traditioneel het configureren, beheren en monitoren van IT systemen met als doel een stabiele en veilige werking te garanderen. Binnen servicegerichte IT organisaties wordt deze rol vaak geplaatst binnen operationele processen zoals incidentbeheer, wijzigingsbeheer en continuïteitsbewaking, zoals beschreven binnen IT-servicemana\- gementkaders \textcite{ITILFoundation}.

In complexe infrastructuren, waaronder mainframe omgevingen, reikt de rol van de System Administrator verder dan louter operationele ondersteuning. Dergelijke omgevingen vereisen diepgaande kennis van systeemarchitectuur, beveiligingsmechanismen en fouttolerantie. \textcite{MainframeReliabilityEnterprise} tonen aan dat mainframes specifiek worden ingezet omwille van hun uitzonderlijke betrouwbaarheid en beschikbaarheid, wat impliceert dat beheerders een cruciale verantwoordelijkheid dragen in het waarborgen van deze eigenschappen.

De literatuur over system administration beschouwt deze rol als een combinatie van technische expertise en operationele verantwoordelijkheid \autocite{NemethEtAl2017}.

Binnen de context van z/OS besturing systeem impliceert dit dat System Administrators niet alleen instaan voor dagelijkse operationele taken, maar ook voor het proactief beheren van prestaties, beveiliging en systeemstabiliteit op lange termijn. De officiële documentatie van IBM Redbooks benadrukt dat het beheer van z/OS-systemen een hoge mate van specialisatie vereist, wat de nood aan duidelijke rolafbakening en competentiedefinitie verder versterkt \autocite{IBMzOSOverview,IBMRedbookzOS}.


\subsection{Competenties en competentieframeworks in IT}

Competenties worden in de literatuur doorgaans gedefinieerd als een samenhangend geheel van kennis, vaardigheden en professionele attitudes die nodig zijn om een bepaalde functie effectief uit te voeren. Binnen IT-organisaties worden competentieframeworks gebruikt om deze elementen te structureren en te koppelen aan rollen, verantwoordelijkheden en opleidingsdoelstellingen \textcite{ITILFoundation}.

Competentieframeworks bieden organisaties ondersteuning bij het evalueren van medewerkers, het uittekenen van opleidingstrajecten en het faciliteren van doorgroei. In de context van complexe IT omgevingen dragen dergelijke frameworks bij tot het expliciteren van verwachtingen en het verminderen van operationele risico’s. Toch blijkt uit de literatuur dat veel bestaande frameworks een generiek karakter hebben en onvoldoende rekening houden met domeinspecifieke kenmerken.


Voor gespecialiseerde infrastructuren zoals mainframe en z/OS-omgevingen leidt dit tot een spanningsveld tussen theoretische competentiemodellen en de praktische realiteit van systeembeheer. De complexiteit en kriticiteit van deze omgevingen vereisen een meer gedifferentieerde benadering van competenties, afgestemd op specifieke verantwoordelijkheden en ervaringsniveaus \autocite{IBMRedbookzOS}.


\subsection{Kennisoverdracht en ervaringsniveaus binnen IT teams}

Binnen IT-teams vormt kennisoverdracht een cruciale factor voor de continuïteit van dienstverlening, zeker in omgevingen waar expertise schaars en sterk gespecialiseerd is. In het bijzonder binnen mainframe contexten wordt deze uitdaging versterkt door de vergrijzing van ervaren profielen en de beperkte instroom van nieuwe specialisten \autocite{MainframeReliabilityEnterprise}.

Het expliciet definiëren van ervaringsniveaus, zoals System Administrator Level~1 en Level~2, kan bijdragen tot een duidelijker onderscheid in verantwoordelijkheden, verwachtingen en opleidingsbehoeften. Dergelijke niveaudifferentiatie ondersteunt niet alleen kennisoverdracht, maar faciliteert ook gerichte coaching en doorgroei binnen teams.

De bestaande literatuur en documentatie bieden echter weinig concrete richtlijnen voor het structureel invullen van deze ervaringsniveaus binnen z/OS-omgevingen. Hierdoor blijven rolverdelingen vaak impliciet en afhankelijk van organisatie-specifieke afspraken, wat de nood aan een gestructureerd competentiekader verder onderstreept \autocite{IBMzOSOverview}.

% Voor literatuurverwijzingen zijn er twee belangrijke commando's:
% \autocite{KEY} => (Auteur, jaartal) Gebruik dit als de naam van de auteur
%   geen onderdeel is van de zin.
% \textcite{KEY} => Auteur (jaartal)  Gebruik dit als de auteursnaam wel een
%   functie heeft in de zin (bv. ``Uit onderzoek door Doll & Hill (1954) bleek
%   ...'')

\subsection{Synthese en positionering van het onderzoek}

Uit de geraadpleegde literatuur blijkt dat mainframe en z/OS-omgevingen gekenmerkt worden door een hoge mate van complexiteit en een sterke afhankelijkheid van gespecialiseerde kennis. De rol van de System Administrator binnen deze context is essentieel voor het waarborgen van stabiliteit, veiligheid en continuïteit van bedrijfskritische systemen.

Hoewel algemene IT kaders en documentatie het belang van competenties, kennisoverdracht en rolafbakening erkennen, ontbreekt een expliciet en gestructureerd competentiemodel dat rekening houdt met verschillende ervaringsniveaus binnen z/OS-omgevingen. Deze leemte bemoeilijkt zowel opleiding als doorgroei van System Administrators.

Deze bachelorproef positioneert zich binnen deze vastgestelde lacune door te focussen op het expliciteren en structureren van kerncompetenties voor System Administrator Level~1 en Level~2 binnen z/OS-omgevingen. Het beoogde resultaat is een toepasbaar competentieframework dat aansluit bij de noden van zowel organisaties als IT professionals.


%---------- Methodologie ------------------------------------------------------
\section{Methodologie}%
\label{sec:methodologie}

Deze bachelorproef hanteert een ontwerpend onderzoeksopzet met als doel het ontwikkelen van een gestructureerd competentieframework voor System Administrators Level~1 en Level~2 binnen z/OS-omgevingen. Het onderzoek combineert literatuurstudie, documentanalyse en de ontwikkeling van een proof of concept. De methodologie bestaat uit vier opeenvolgende fasen.

\subsection{Literatuurstudie}

In een eerste fase wordt een uitgebreide literatuurstudie uitgevoerd naar mainframe omgevingen, het beheer van z/OS-systemen en bestaande competentie- en rolframeworks binnen IT. Hierbij wordt gebruikgemaakt van wetenschappelijke artikels, professionele vakliteratuur en officiële documentatie, zoals IBM Redbooks en IT-serviceman\-
agementkaders.

Deze literatuurstudie vormt de basis voor het identificeren van relevante competentiedomeinen en biedt inzicht in bestaande benaderingen van rolafbakening en ervaringsniveaus binnen IT-functies.

\subsection{Analyse van bestaande rol en competentiemodellen}

In de tweede fase worden bestaande rolbeschrijvingen en competentiemodellen geanalyseerd, zoals ITIL roldefinities en andere relevante industriestandaarden. Deze analyse richt zich op het identificeren van taken, verantwoordelijkheden en competenties die relevant zijn voor system administrators in complexe IT-omgevingen.

De focus ligt hierbij op het vergelijken van generieke modellen met de specifieke kenmerken van mainframe- en z/OS-omgevingen, met bijzondere aandacht voor het onderscheid tussen junior- en meer ervaren profielen.

\subsection{Ontwerp van het competentieframework}

Op basis van de bevindingen uit de literatuurstudie en de analysefase wordt een conceptueel competentieframework opgesteld voor System Administrator Level~1 en Level~2 binnen z/OS-omgevingen. Dit framework structureert competenties volgens verschillende domeinen, waaronder technische kennis, operationele vaardigheden en professionele verantwoordelijkheden.

Het framework wordt ontworpen met het oog op toepasbaarheid binnen organisaties en bruikbaarheid voor opleiding, onboarding en doorgroei, zonder in detail in te gaan op specifieke tools of bedrijfsspecifieke processen.

\subsection{Proof of Concept (PoC): website}

Als laatste fase wordt een proof of concept ontwikkeld in de vorm van een website waarin het competentieframework op een overzichtelijke en toegankelijke manier wordt voorgesteld. De website bevat theoretische toelichting per competentiedomein, gekoppeld aan het onderscheid tussen Level 1 en Level 2.

Daarnaast worden voorbeeldscenario’s en oefenvormen opgenomen om de praktische toepasbaarheid van het framework te illustreren binnen een opleidingscontext. Deze proof of concept dient als validatie van het framework en toont aan hoe het model kan worden ingezet binnen de praktijk.

\subsection{Gebruikte tools en technologieën}

Voor de uitvoering van dit onderzoek worden verschillende tools ingezet. Voor literatuurbeheer wordt gebruikgemaakt van JabRef in combinatie met \LaTeX{} en Bib\LaTeX{}. De proof of concept website wordt ontwikkeld met behulp van webtechnologieën zoals HTML, CSS en JavaScript.

De website wordt gehost via een publieke repository op GitHub en publiek toegankelijk gemaakt met behulp van GitHub Pages. Op deze manier kan het ontwikkelde competentieframework eenvoudig gedeeld worden en blijft het resultaat transparant en reproduceerbaar.

\subsection{Planning en deliverables}

Het onderzoek wordt uitgevoerd over een periode van meerdere maanden. In de eerste fase wordt de literatuurstudie afgerond. Vervolgens worden bestaande rol en competentiemodellen geanalyseerd. De derde fase resulteert in een uitgewerkt competentieframework. In de laatste fase wordt de proof of concept website ontwikkeld en gedocumenteerd.

De belangrijkste deliverables van deze bachelorproef zijn de bachelorproeftekst waarin het competentieframework wordt uitgewerkt en onderbouwd, en een proof of concept website die dit framework op een toegankelijke en praktische manier visualiseert.


%---------- Verwachte resultaten ----------------------------------------------
\section{Verwachte resultaten en conclusie}%
\label{sec:verwachte_resultaten}

Het verwachte resultaat van deze bachelorproef is de ontwikkeling van een gestructureerd en toepasbaar competentieframework voor System Administrators Level~1 en Level~2 binnen z/OS-omgevingen. Dit framework brengt de kerncompetenties in kaart die nodig zijn om deze rollen effectief te kunnen invullen, met aandacht voor technische kennis, operationele vaardigheden en professionele verantwoordelijkheden.

Daarnaast wordt een proof of concept ontwikkeld in de vorm van een website waarin het competentieframework op een overzichtelijke en toegankelijke manier wordt voorgesteld. Deze website fungeert als praktische vertaling van het theoretische framework en illustreert hoe het model kan worden ingezet binnen een opleidings- of onboardingcontext.

Er wordt verwacht dat het voorgestelde framework bijdraagt aan een duidelijkere rolafbakening tussen System Administrator Level~1 en Level~2, wat organisaties kan ondersteunen bij het structureren van opleidingstrajecten en doorgroeipaden. Voor (startende) System Administrators kan het framework fungeren als referentiepunt voor professionele ontwikkeling en het verwerven van domeinspecifieke kennis binnen z/OS-omgevingen.

De meerwaarde van deze bachelorproef situeert zich in het expliciteren van impliciete kennis binnen een gespecialiseerd IT domein. Door competenties te structureren en toegankelijk te maken, kan het onderzoek bijdragen aan kennisoverdracht, continuïteit van beheer en het verminderen van afhankelijkheid van individuele expertise. Hoewel de concrete impact afhankelijk is van de organisatiecontext waarin het framework wordt toegepast, wordt verwacht dat het ontwikkelde model voldoende flexibel is om als fundament te dienen voor verdere verfijning en praktische toepassing.

Tot slot biedt deze bachelorproef een basis voor toekomstig onderzoek, bijvoorbeeld door het framework verder te valideren in een bedrijfscontext of uit te breiden naar andere ervaringsniveaus of gespecialiseerde rollen binnen mainframe omgevingen.



\printbibliography[heading=bibintoc]

\end{document}